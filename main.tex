%%%%%% CREDIT TO MICHAEL PHAM FOR THIS TEMPLATE

\documentclass{article}
%%%%%%% PREAMBLE %%%%%%%
%BEGIN_FOLD
%%%%% PACKAGES
\usepackage{amsmath}
\usepackage{amssymb}
\usepackage{amsthm}
\usepackage{cabin} % section title font
\usepackage[sfdefault,condensed]{roboto}
\usepackage[T1]{fontenc}
\usepackage[shortlabels]{enumitem}
\usepackage{fancyhdr}
\usepackage{graphicx}
\usepackage{hyperref}
\usepackage{mathtools}
\usepackage[framemethod=TikZ]{mdframed}
\usepackage[scr]{rsfso} % power set symbol
\usepackage{tasks} % vaguely remember this being important for something...?
\usepackage{tikz} % diagrams
\usepackage{titlesec}
\usepackage{thmtools}
\usepackage{varwidth}
\usepackage{verbatim} % longer comments
\usepackage{xcolor}
%%%%%

%%%%% COLOURS
\definecolor{darkgreen}{HTML}{19A514}
\definecolor{lightgreen}{HTML}{9DFF9A}
\definecolor{darkblue}{HTML}{3E5FE4}
\definecolor{lightblue}{HTML}{BCDEFF}
\definecolor{darkred}{HTML}{CC3333}
\definecolor{lightred}{HTML}{FFA9A9}
\definecolor{darkpurple}{HTML}{A933CD}
\definecolor{lightpurple}{HTML}{F0BAFF}
\definecolor{darkyellow}{HTML}{D2D22A}
\definecolor{lightyellow}{HTML}{FFFFAE}
\definecolor{hyperlinkblue}{HTML}{3366CC}
%%%%%

%%%%% PAGE SETUP
% BASIC %
%\setlength\parindent{0pt} % paragraph indentation
\setlength{\parskip}{5pt} % spacing between paragraphs
\usepackage[margin=1in]{geometry} % margin size
\setcounter{section}{-1}

% HEADER/FOOTER %
\pagestyle{fancy}
\fancyhf{}
\fancyfoot[R]{\thepage} % page number on bottom right
\fancyhead[R]{\textit{\leftmark}} % section title
\renewcommand{\headrulewidth}{0pt} % removing horizontal line at the top

% HYPERLINK FORMATTING %
\hypersetup{
	colorlinks,    
	linkcolor=hyperlinkblue,
	urlcolor=hyperlinkblue,
	pdftitle={...},
	pdfauthor={Daniel Liu},
}

%%%%%

%%%%% ENVIRONMENTS STYLES
% SOLUTION ENVIRONMENT %
\newenvironment{solution}{\begin{proof}[Solution]}{\end{proof}}

% PURPLE BOX %
\declaretheoremstyle[
mdframed={
	backgroundcolor=lightpurple,
	linecolor=darkpurple,
	rightline=false,
	topline=false,
	bottomline=false,
	linewidth=2pt,
	innertopmargin=8pt,
	innerbottommargin=8pt,
	innerleftmargin=8pt,
	leftmargin=-2pt,
	skipbelow=2pt,
},
headfont=\normalfont\bfseries\color{darkpurple}
]{purplebox}

% GREEN BOX %
\declaretheoremstyle[
mdframed={
	backgroundcolor=lightgreen,
	linecolor=darkgreen,
	rightline=false,
	topline=false,
	bottomline=false,
	linewidth=2pt,
	innertopmargin=8pt,
	innerbottommargin=8pt,
	innerleftmargin=8pt,
	leftmargin=-2pt,
	skipbelow=2pt,
},
headfont=\normalfont\bfseries\color{darkgreen}
]{greenbox}

% YELLOW BOX %
\declaretheoremstyle[
mdframed={
	backgroundcolor=lightyellow,
	linecolor=darkyellow,
	rightline=false,
	topline=false,
	bottomline=false,
	linewidth=2pt,
	innertopmargin=8pt,
	innerbottommargin=8pt,
	innerleftmargin=8pt,
	leftmargin=-2pt,
	skipbelow=2pt,
},
headfont=\normalfont\bfseries\color{darkyellow}
]{yellowbox}

% BLUE BOX %
\declaretheoremstyle[
mdframed={
	backgroundcolor=lightblue,
	linecolor=darkblue,
	rightline=false,
	topline=false,
	bottomline=false,
	linewidth=2pt,
	innertopmargin=8pt,
	innerbottommargin=8pt,
	innerleftmargin=8pt,
	leftmargin=-2pt,
	skipbelow=2pt,
},
headfont=\normalfont\bfseries\color{darkblue}
]{bluebox}

% RED BOX %
\declaretheoremstyle[
mdframed={
	backgroundcolor=lightred,
	linecolor=darkred,
	rightline=false,
	topline=false,
	bottomline=false,
	linewidth=2pt,
	innertopmargin=8pt,
	innerbottommargin=8pt,
	innerleftmargin=8pt,
	leftmargin=-2pt,
	skipbelow=2pt,
},
headfont=\normalfont\bfseries\color{darkred}
]{redbox}
%%%%%

%%%%% ENVIRONMENTS
% PURPLE BOXES (theorems, propositions, lemmas, and corollaries) %
\declaretheorem[style=purplebox,name=Theorem,within=section]{thm}
\declaretheorem[style=purplebox,name=Theorem,sibling=thm]{theorem}
\declaretheorem[style=purplebox,name=Theorem,numbered=no]{thm*, theorem*}
\declaretheorem[style=purplebox,name=Proposition,sibling=thm]{prop, proposition}
\declaretheorem[style=purplebox,name=Proposition,numbered=no]{prop*, proposition*}
\declaretheorem[style=purplebox,name=Lemma,sibling=thm]{lem, lemma}
\declaretheorem[style=purplebox,name=Lemma,numbered=no]{lem*, lemma*}
\declaretheorem[style=purplebox,name=Corollary,sibling=thm]{cor, corollary}
\declaretheorem[style=purplebox,name=Corollary,numbered=no]{cor*, corollary*}

% GREEN BOXES (definitions) %
\declaretheorem[style=greenbox,name=Definition,sibling=thm]{definition, defn}
\declaretheorem[style=greenbox,name=Definition,numbered=no]{definition*, defn*}
\declaretheorem[style=greenbox,name=Property,numbered=no]{property}

% BLUE BOXES (problems) %
\declaretheorem[style=bluebox,name=Problem,numberwithin=section]{homework, hw}
\declaretheorem[style=bluebox,name=Problem,numbered=no]{homework*, hw*}

% RED BOXES %
\declaretheorem[style=redbox,name=Remark,sibling=thm]{remark, rmk}
\declaretheorem[style=redbox,name=Remark, numbered=no]{remark*, rmk*}
\declaretheorem[style=yellowbox,name=Warning,sibling=thm]{warn}
\declaretheorem[style=yellowbox,name=Warning,numbered=no]{warn*}
%%%%%

%%%%% PROOF FORMATTING
\renewcommand\qedsymbol{$\blacksquare$}
%%%%%

%%% CUSTOM COMMANDS
% basic %
\newcommand{\Mod}[1]{\ (\mathrm{mod}\ #1)}
\newcommand{\floor}[1]{\left\lfloor{#1}\right\rfloor}
\newcommand{\ceil}[1]{\left\lceil{#1}\right\rceil}
\newcommand{\norm}[1]{\left\lVert{#1}\right\rVert}

% logic %
\newcommand*\xor{\oplus}
\newcommand{\all}{\forall}
\newcommand{\bland}{\bigwedge}
\newcommand{\blor}{\bigvee}
\newcommand*{\defeq}{\mathrel{\rlap{\raisebox{0.3ex}{$\m@th\cdot$}}\raisebox{-0.3ex}{$\m@th\cdot$}}=} \makeatother

% matrices %
\newcommand\aug{\fboxsep=- \fboxrule\!\!\!\fbox{\strut}\!\!\!}\makeatletter 

% sets %
\newcommand{\CC}{\mathbb{C}}
\newcommand{\NN}{\mathbb{N}}
\newcommand{\QQ}{\mathbb{Q}}
\newcommand{\RR}{\mathbb{R}}
\newcommand{\ZZ}{\mathbb{Z}}

% probability stuff %
\newcommand{\E}{\mathbb{E}}
\newcommand{\Var}{\mathrm{Var}}
\newcommand{\var}{\mathrm{Var}}
\newcommand{\cov}{\mathrm{cov}}
\newcommand{\corr}{\mathrm{Corr}}

% title %
\newcommand{\mytitle}[2]{%
	\title{#1}
	\author{Daniel Liu}
	\date{#2}
	\maketitle
	\newpage
	\tableofcontents
	\newpage
}
%%%
%%%%%
%END_FOLD
%%%%%


\begin{document}
\author{Daniel Liu}
\mytitle{CS170 Notes}{}

\section{Prologue}

\subsection{Fibonacci sequence}
\begin{definition}[Fibonacci Sequence]
\[ F_n =
  \begin{cases} F_{n-1} + F_{n-2} & \text{if $n > 1$} \\
    1 & \text{if $n = 1$} \\
    0 & \text{if $n = 0$}
  \end{cases}
\]
\end{definition}
The naive approach is to return \verb|fib1(n-1) + fib1(n-2)|. Let $T(n)$ be the number of computer steps to compute \verb|fib1(n)|. We know that:
\begin{itemize}
  \item $T(n) \leq 2$ for $n \leq 1$.
  \item $T(n) = T(n-1) + T(n-2) + 3$ for $n > 1$. There are three computer steps, two checks and one addition.
\end{itemize}
$T(n)$ is therefore exponential in $n$, so the algorithm is impractically slow. \par
We can rewrite the code in polynomial time. First, create an array of size $n$, and set \verb|arr[0] = 0, arr[1] = 1| and do \verb|arr[i] = arr[i - 1] + arr[i -2]| in a linear for loop.
This algorithm is linear in $n$, which is a reasonable runtime.

\subsection{Big-$O$ notation}
We will express runtime as a function of the size of the input.
\begin{itemize}
  \item $O$ is the \emph{upper bound}, an analog of $f \leq g$. $f = O(g)$ if $\exists c > 0, f(n) \leq c \times g(n)$
  \item $\Omega$ is the \emph{lower bound}, an analog of $f \geq g$. $f = \Omega(g)$ if $\exists c > 0, f(n) \leq c \times g(n)$
  \item $\Theta$ is the \emph{tightest bound}, an analog of $f = g$. $f = \Theta(g)$ if $f = O(g)$ and $f = \Omega(g)$
\end{itemize}
We do not just use the total number of steps that a computer runs; there are a few rules that are used to further simplify.
\begin{enumerate}
  \item Multiplicative constants are omitted. For example, $14n^2$ becomes $n^2$.
  \item $n^a$ dominates $n^b$ if $a > b$. For example, $n^2$ dominates $n$.
  \item Any exponential dominates any polynomial. For example, $3^n$ dominates $n^5$.
  \item Any polynomial dominates any logarithm. For example, $n$ dominates $(\log n)^3$
\end{enumerate}
\newpage

\section{Algorithm with numbers}

\subsection{Basic Arithmetic}
\subsubsection{Addition}
It is a basic rule that the sum of any three single digit numbers is at most two digits long.
This rule allows us to define a method of addition that works in any base, since we know each carry is always a single digit and at any step at most three single digit numbers are added. \par
With $k$ digits in base $b$, we can express numbers up to $b^k - 1$.
To represent the number $N \geq 0$ in base $b$, we find that $\ceil{\log_b(N+1)}$ digits are needed to write $N$ in base $b$.
To convert logarithms from base $a$ to base $b$, we use the equation $\log_b N = (log_a N)/(log_a b)$.
The size of an integer $N$ in base $a$ is its size in base $b$ times a constant factor $\log_a b$.
Therefore, the size of an integer in a different base in big-$O$ notation is simple $O(\log N)$. \par
We want to find out how long it takes to add two binary numbers $x$ and $y$.
Suppose $x$ and $y$ are both $n$ bits long, then the sum of $x$ and $y$ is at most $n + 1$ bits, and each individual bit is constant time.
The total runtime of the algorithm is linear and in the form $c_0 + c_1 n$, where $c_0$ and $c_1$ are some constant.

\subsection{Multiplication and division}
The traditional human method of multiplication takes $O(n^2)$ runtime.
Another method is to line up the two numbers $x$ and $y$, floor divide $x$ by $2$ and double $y$.
Keep going until $x = 1$.
Then strike out all of the rows where $x$ is even, and add the remaining $y$s.
This method can be represented by the following recursive formula:
\[x \cdot y = \begin{cases}
  2(x \cdot \floor{y / 2}) & \text{if $y$ is even} \\
  x + 2(x \cdot \floor{y / 2}) & \text{if $y$ is odd} \\
\end{cases}\]
Each recursive call requires a division by $2$ (right shift), a test for odd/even, a multiplication by $2$ (left shift), and possibly one addition, which is $O(n)$ bit operations.
The algorithm terminates after $O(n)$ recursive calls, since each call reduces the number of bits by $1$.
Therefore, the total runtime of this algorithm is $O(n^2)$.\par
For division, to divide an integer $x$ by another integer $y \neq 0$ means to find a quotient $q$ and a remainder $r$ where $x = yq + r$ and $r < y$.

\subsection{Modular Arithmetic}
Modular arithmetic deals with restricted ranges of integers. 
$x$ \emph{modulo} $N$ is the remainder when $x$ is divided by $N$. 
If $x = qN + r$, with $0 \leq r < N$, then $x \mod{N} = r$.
\begin{prop}[Substitution Rule]
  If $x \equiv x' \mod{N}$ and $y \equiv y' \mod{N}$, then \\
  $x + y \equiv x' + y' \mod{N}$ and $xy \equiv x'y' \mod{N}$
\end{prop}
Additionally, Associativity, Commutativity, and Distributivity all hold for modular arithmetic.

\subsubsection{Modular Addition and Multiplication}
To add two numbers $x$ and $y$ modulo $N$, we start with regular addition.
If the sum exceeds $N-1$, we just need to subtract $N$. 
Therefore, the runtime is $O(n)$, where $n = \ceil{\log N}$.\par
To multiply two mod-$N$ numbers, $x$ and $y$, we can use regular multiplication and reduce the answer modulo $N$.
The resulting product can be as large as $(N-1)^2$, but this is still at most $2n$ bits long.
We then can compute the remainder by dividing it by $N$, which takes $O(n^2)$.

\subsubsection{Modular Exponentiation}
We wish to calculate $x^y \mod{N}$ for large values of $x$, $y$, and $N$.
To ensure the numbers don't grow too large, we can repeatedly multiply by $x \mod{N}$ until reaching $x^y \mod{N}$.
\[x \mod{N} \rightarrow x^2 \mod{N} \rightarrow \dots \rightarrow x^y \mod N\]
This algorithm is exponential in the size of $y$. \par
We can do better by starting with $x$ and squaring repeatedly modulo $N$.
\[x \mod{N} \rightarrow x^2 \mod{N} \rightarrow x^4 \mod N \rightarrow \dots \rightarrow x^{2 \floor{\log y}} \mod N\]
Each square takes $O(\log^2 N)$ to compute, with $\log y$ multiplications.
To determine $x^y \mod N$, we need to separate $y$ into its powers of $2$.
For example:
\[x^{25} = x^{16} \cdot x^8 \cdot x^1\]
Rewriting this idea recursively:
\[ x^y = 
\begin{cases}
  (x^{\floor{y/2}})^2 & \text{if $y$ is even} \\
  x \cdot (x^{\floor{y/2}})^2 & \text{if $y$ is odd} \\
\end{cases}
\]
Let $n$ be the size in bits of $x$, $y$, and $N$ (whichever is largest).
The algorithm halts after at most $n$ recursive calls, multiplying $n$ bit numbers every call.
Therefore the runtime of exponentiation is $O(n^3)$

\subsubsection{Euclid's Algorithm for Greated Common Divisor}
Given two integers $a$ and $b$, we want to find the largest integer that divides both of them, also known as the \emph{Greatest Common Divisor}.
\begin{thm}[Euclid's Rule]
  If $x$ and $y$ are positive integers with $x \geq y$, then $gcd(x, y) = gcd(x \mod y, y)$.
\end{thm}
\begin{proof}
  If we can show $\gcd(x, y) = \gcd(x-y, y)$, above statement can be derived by repeatedly subtracting $y$ from $x$.
  Any integer that divides $x$ and $y$ must divide $x-y$, so $\gcd(x, y) \leq \gcd(x - y, y)$.
  Any integer that divides $x-y$ and $y$ must divide both $x$ and $y$, so $\gcd(x, y) \geq \gcd(x-y, y)$.
\end{proof}

\begin{lemma}
  If $a \geq b$, then $a \mod b < a / 2$.
\end{lemma}
\begin{proof}
	If $b \leq a/2$ then $a \mod b < b \leq a/2$, else $b > a/2$ and $a \mod b = a - b \leq a/2$.
\end{proof}
This means that after any two consecutive rounds of Euclid's rule, $a$ and $b$ are at least halved in value or their length decreases by one bit.
Two $n$-bit integers will reach the base case by $2n$ calls.
This gives a total runtime of $O(n^3)$, since each call takes quadratic-time for division, and there are at most $2n$ calls.

\subsubsection{An Extension of Euclid's Algorithm}
\begin{lemma}
  If $d$ divides $a$ and $b$, and $d = ax + by$ for $\exists x, y$, then $d = \gcd(a, b)$
\end{lemma}
\begin{proof}
	  $d$ is a common divisor of $a$, $b$, so $d \leq \gcd(a, b)$. Since $\gcd(a, b)$ is common divisor of $a$, $b$, it must divide $ax + by = d$, which implies $gcd(a, b) \leq d$. Therefore $\gcd(a, b) \leq d$
\end{proof}

\begin{lemma}
  For any positive integers $a$ and $b$, the extended Euclid algorithm returns integers $x, y, d$ such that $\gcd(a, b) = d = ax + by$
\end{lemma}
\begin{proof}
We attempt to prove this by induction.
\begin{enumerate}
  \item Base case $b = 0$, returns $a$ which is $\gcd(a, b)$.
	\item Inductive hypothesis: Assume $\gcd(b, a \mod b) = bx' + (a \mod b) y'$.
  \item Inductive step: $(a \mod b) = (a - \lfloor a/b \rfloor b)$, $d = \gcd(a, b) = \gcd(b, a \mod b) = bx' + (a \mod b) y' + bx' + (a - \lfloor a/b \rfloor b) y' = ay' + b(x' - \lfloor a/b \rfloor y')$. Therefore $d = ax + by$ with $x = y'$ and $y = x' - \lfloor a/b \rfloor y'$
\end{enumerate}
\end{proof}

\subsubsection{Modular Division}
We say that $x$ is the multiplicative inverse of $a \mod N$ if $ax \equiv 1 \mod N$.
There can be at most one $x \mod N$ that satisfies this condition, denoted by $a^{-1}$.
However, the multiplicative inverse may not always exist. For example, if $a$ and $N$ are both even, $a \mod N = a - kN$, so the inverse does not exist.
Because $\gcd(a, N)$ divides $ax \mod N$, then if $gcd(A, N) > 1$, $ax \not\equiv 1 \mod N$
When $\gcd(a, N) = 1$, the extended Euclid can give us $x, y$ where $ax + Ny = 1$, so $ax \equiv 1 \mod N$.
\begin{thm}
  For any $a \mod N$, $a$ has a multiplicative inverse if and only if it is relatively prime to $N$.
  If the inverse exists, it can be found in $O(n^3)$ time by running the extended Euclid algorithm.
\end{thm}

\subsection{Primality Testing}
Factoring is hard, but determining primality is easy.
We hope to find some test that determines if a number is prime without trying to factor the number.
\begin{thm}[Fermat's Little Theorem]
  If $p$ is prime, then for every $1 \leq a < p$, $a^{p-1} \equiv 1 \mod p$
\end{thm}
\begin{proof}
  Let $S$ be the nonzero integers modulo $p$, $S = \{1, 2, \dots, p-1\}$.
  If we multiply the numbers in $S$ by $a \mod p$, we just permute them.
  This is because none of the numbers can equal each other when multiplying by $a \mod p$, since they are unique from the start.
  Therefore, $S = \{1, 2, \dots, p-1\} = \{a \cdot 1 \mod p, a \cdot 2 \mod p, \dots, a \cdot (p - 1) \mod p\}$.
  Multiplying all the elements together, we get $(p-1)! \equiv a^{p-1} \cdot (p-1)! \mod p$.
  Dividing by $(p-1)!$, we get the theorem.
\end{proof}
This proof suggests that if $a^{N-1} \equiv 1 \mod N$, for $a < N$, then $N$ is prime.
However, Fermat's Theorem is not if and only if, so it does not work with certain choices of $a$.

\begin{lemma}
  If $a^{N-1} \not\equiv 1 \mod N$ for $a$ relatively prime to $N$, it must hold for at least half the choices of $a < N$
\end{lemma}
\begin{proof}
  Fix some value for $a$ which $a^{N-1} \not \equiv 1 \mod N$.
  Every element $b < N$ that passes Fermat's test with respect to $N$ has a twin, $a \cdot b$ that fails the test: $(a \cdot b) ^ {N-1} \equiv a^{N-1} \cdot b^{N-1} \equiv a^{N-1} \not \equiv 1 \mod N$.
  All elements $a \cdot b$ for a fixed $a$ but different $b$ are distinct, so there is a one-to-one function $b \mapsto a \cdot b$, where there are at least as many elements which fail the test as pass it.
\end{proof}
We can then test primality with a low error probability by selecting $k$ numbers less than $N$, then testing each one with $N$.
The probability that the algorithm returns yes when $N$ is not prime is $\leq \frac{1}{2^k}$.

\subsubsection{Generating Random Primes}
Primes are very common.
A random $n$ bit number has around $\frac{1}{n}$ chance of being prime.
\begin{thm}[Lagrange's Prime Number Theorem]
  Let $\pi(x)$ be number of primes $\leq x$.
  Then $\pi(x) \approx x/(\ln x)$, or $\lim_{x \rightarrow \infty} \frac{\pi(x)}{x/ \ln x} = 1$
\end{thm}
To generate a random $n$-bit prime, we can just pick a random $n$-bit number $N$, run the primality test on it, then output $N$ if it passes the test else repeat the process.
\begin{definition}[Carmichael numbers]
There are non-prime numbers that can fool Fermat tests for all $a$, known as \emph{Carmichael numbers}. We can use the following algorithm to check if $N$ is composite.
\begin{itemize}
  \item Write $N-1$ in the form $2^tu$.
  \item Choose a random base $a$ and check the value of $a^{N-1} \mod N$ by starting at $a^u \mod N$ and repeatedly squaring.
  \item If $a^{N-1} \not\equiv 1 \mod N$, then $N$ is composite.
  \item Otherwise somewhere in the test we must have run into a $1$. If this is after the first position and the preceding value was not $-1$, then $N$ is composite.
\end{itemize}
\end{definition}
Combining with the earlier Fermat test, at least three-fourths of the possible values of $a$ between $1$ and $N-1$ will reveal a composite $N$, even with Carmichael numbers

\subsection{Cryptography}
Suppose Alice wants to send a message to Bob, but Eve can intercept the message and try to decrypt it.

\subsubsection{Private-Key Schemes: One-time pad and AES}
The encryption function $e$ is an invertible function whose inverse is the decryption function $d$.
\begin{definition}[One-time Pad]
  Alice and Bob must meet beforehand to choose a binary string $r$ with $n$ bits.
  Alice's encryption is the bitwise exclusive-or, or $e_r(x) = x \oplus r$.
  $e_r(x)$ is its own inverse: $e_r(e_r(x)) = (x \oplus r) \oplus r = x \oplus 0 = x$
  To be secure, we choose $r$ at random, so Eve can get no information about $x$ without knowing $r$.
  The downside to this method is that it is only a one-time use, since sending another message gives Eve $x_1 \oplus x_2$, which could reveal important information such as repeated bits.
\end{definition}

\subsubsection{RSA}
  In public-key cryptography, anybody can send a message to anybody else using publically available information.

\begin{thm}
  Pick two primes, $p$ and $q$, and let $N = pq$. 
  For any $e$ relatively prime to $(p-1)(q-1)$ and $d = e^{-1} \mod (p - 1)(q - 1)$, the following is true:
  \begin{itemize}
    \item The mapping $x \mapsto x^e \mod N$ is a biject on $\{0, 1, \dots N-1 \}$
    \item $\forall x \in \{0, \dots, N-1\}$, $(x^e)^d \equiv x \mod N$.
  \end{itemize}
\end{thm}
\begin{proof}
  If the mapping $x \mapsto x^e \mod N$ is invertible, it must be a bijection, so statement 2 implies statement 1.
  To prove statement 2, $e$ is invertible modulo $(p-1)(q-1)$ since it is relatively prime.
  Since $ed \equiv 1 \mod (p-1)(q-1)$, $\exists k: ed \equiv 1 + k(p - 1)(q - 1)$.
  We can show $x^{1 + k(p-1)(q-1)} - x \equiv 0 \mod N$ by Fermat's little theorem, since $x^{p-1} \equiv 1 \mod p$ and $x^{q-1} \equiv 1 \mod q$, so this is also divisible by $N$.
\end{proof}

\subsection{Universal Hashing}
\subsubsection{Hashing}
Hashing is a method of storing data items in a table to support insertions, deletions, and lookups.
We give a short "nickname" to each of the items, and each item with the same "nickname" stored in a linked list.

\subsubsection{Hash Functions}
It is difficult to create a good hash function, since it must be "Random" enough to scatter data round but "consistent" enough to get the same result every time.
\begin{definition}
Let the number of buckets be a prime number.
For any four coefficients $a_1, \dots, a_4 \in \{0, 1, \dots, n-1\}$, define $h_a(x_1, \dots, x_4) = \sum_{i=1}^{4} a_i \cdot x_i \mod n$
\end{definition}

\begin{thm}
  Consider any pair of distinct IP addresses $x=(x_1, \dots, x_4)$ and $y=(y_1,\dots,y_4)$.
  If the coefficients $a=(a_1, a_2, a_3, a_4)$ are chosen uniformly at random from $\{0, 1, \dots, n-1\}$, then $\Pr\{h_a(x_1, \dots, x_4) = h_a(y_1, \dots, y_4)\} = \frac{1}{n}$.
	In other words, the chance that $x$ and $y$ collide under $h_a$ is the same as it would be if each were assigned nicknames randomly and independently
\end{thm}
\begin{proof}
  Since $x=(x_1, \dots, x_4)$ and $y=(y_1, \dots, y_4)$ are distinct, the quadruples must differ in some component.
  Assume that $x_4 \not = y_4$.
  We wish to compute the probability $\Pr\{h_a(x_1, \dots, x_4) = h_a(y_1, \dots, y_4)\}$, or that the probability $\sum_{i=1}^4 a_i \cdot x_i \equiv \sum_{i=1}^4 a_i \cdot y_i \mod n$. This can be rewritten as $\sum_{i=1}^3 a_i \cdot (x_i - y_i) \equiv a_4 \cdot(y_4 - x_4) \mod n$ Looking for probability that the last drawn number $a_4$ is such that the above equation holds. If the left-hand side evaluates to $c$, $a_4$ must be $c \cdot (y_4 - x_4)^{-1} \mod n$ out of $n$ possible values, which is with probability $1/n$.
\end{proof}
Let $\mathcal{H} = \{h_a : a \in \{0, \dots, n-1\}^4\}$, so $\mathcal{H}$ is a random hash function from our family.
For any two distinct data items $x$ and $y$, exactly $|\mathcal{H}|/n$ of all hash functions in $\mathcal{H}$ map $x$ and $y$ to the same bucket, where $n$ is the number of buckets. A family of hash functions with this property is called universal.
This means that for any two data items $x$ and $y$, the chance of collision is the same as if we map $x$ and $y$ to buckets uniformly at random.
We can apply this any general table.
First choose a table size $n$ to be some prime number that is larger than the number of expected items in the table.
Next assume the size of the domain of all items is $N = n^k$, a power of $n$.
Then each data item can be consider a $k$-tuple of integers modulo $n$, and $\mathcal{H} = \{h_a : a \in \{0, \dots, n-1\}^n\}$ is a universal family of hash functions.
\newpage

\section{Divide and Conquer Algorithms}
Divide and Conquer refers to a set of algorithms which involve breaking a problem into subproblems that are smaller instances of the same type of problem, recusively solving the subproblems, and appropriately combining the answers.

\subsection{Multiplication}
We can apply this idea to regular multiplication. Let $x$ and $y$ be two $n$-bit integers, with $n$ being a power of $2$. If $x$ and $y$ are written in binary, we can split them both in half and rewrite as $2^{n/2}x_L + x_R$, $2^{n/2}y_L + y_R$. Then $xy = (2^{n/2}x_L + x_R)(2^{n/2}y_L + y_R) = 2^n x_L y_l + 2^{n/2}(x_L y_R + x_R y_L) + x_R y_R$. This algorithm requires four recursive calls for $x_L y_L$, $x_L y_R$, $x_R y_L$, $x_R y_R$, so $T(n) = 4T(n/2) + O(n)$, which is a runtime of $O(n^2)$. \par
To optimize this, we can exchange $(x_L y_R + x_R y_L) = (x_L + x_R)(y_L + y_R) - x_L y_L - x_R y_R$ for three multiplications instead. This only requires three recursive calls, so $T(n) = 3T(n/2) + O(n)$, giving runtime of $O(n^{1.59})$. The height of the tree is $\log_2 n$, the branching factor is $3$, so at depth $k$ there are $3^k$ subproblems. For each subproblem it takes a linear amount of work to combine answers. Therefore, the total time at depth $k$ is $3^k \times O(\frac{n}{2^k}) = (\frac{3}{2})^k \times O(n)$. The work that is done increases geometrically by a factor of $3/2$ per level, so the overall runtime is $O(n^{\log_2 3}) \approx O(n^{1.59})$. We can see that a small difference in speed per level.

\subsection{Recurrence Relations}
\begin{thm}[Master Theorem]
  If $T(n) = aT(\lceil n/b \rceil) + O(n^d)$ for $a>0, b>0, d \geq 0$, then
$$T(n) =
\begin{cases}
  O(n^d) & \text{if $d > \log_b a$} \\
  O(n^d \log_b n) & \text{if $d = \log_b a$} \\
  O(n^{\log_b a}) & \text{if $d < \log_b a$}
\end{cases}$$
\end{thm}
\begin{proof}
Assume that $n$ is a power of $b$ for convenience. Notice the size of the subproblems decrease by a factor of $b$ with each level, so the height is $\log_b n$ levels. The branching factor is $a$, so the $k$th level is made of $a^k$ subproblems, each with size $n/b_k$. The total work done for a level is $a^k \times O(\frac{n}{b^k})^d = O(n^d) \times (\frac{a}{b^d})^k$. As $k$ goes from $0$ to $\log_b n$, the numbers form a geometric series with ratio $a/b^d$.
\begin{itemize}
	\item If the ratio is less than $1$, the series is decreasing and the sum is $O(n^d)$.
	\item If the ratio is greater than $1$, the series is increasing and the sum is given by last tern $O(n^{\log_b a})$.
	\item If the ratio is exactly $1$, all $O(\log n)$ terms are equal to $O(n^d)$.
\end{itemize}
\end{proof}

\subsection{Mergesort}
The process of mergesort involves splitting the list into two halves, recursively sorting each half, then merging the two sorted sublists.
Merges are linear, so the overall time taken by mergesort is $T(n) = 2T(n/2) + O(n) = O(n \log n)$ \par
There is a $n \log n$ lower bound for sorting with comparisons.
We can depict sorting algorithms as a tree, with each node representing a comparison and the leaves representing a permutation of the elements.
There are therefore $n!$ leaves, so the optimal depth and runtime is $\log(n!)$.
In the worse case, there are $\Omega(n \log n)$ comparisons, so it follows that mergesort is an optimal comparison-based sorting algorithm.

\subsection{Medians}
We want to find the median of a list of numbers.
We can generalize this problem take in a list of numbers, $S$, and integer $k$, then return the $k$th smallest element of $S$.
The median is the output if we input $k = \floor{|S| / 2}$. \par
The divide and conquer strategy for this problem is as follows:
\begin{itemize}
  \item For any number $v$, split $S$ into three categories: $S_L$, $S_V$, $S_R$, which are lists of numbers less than, equal to, or greater than $v$.
  \item The search can be narrowed down to one of the sublists, since we know the size of each sublist and which element we want.
\end{itemize}
    \[
\text{selection}(S, k) = \begin{cases} \text{selection}(S_L, k) & \text{if $k \leq |S_L|$} \\ v & \text{if $S_L < k \leq |S_L| + |S_v|$} \\ \text{selection}(S_R, k - |S_L| - |S_v|) & \text{if $k > |S_L| + |S_v|$} \end{cases}
    \]
The worse case efficiency for this algorithm is $\theta(n^2)$, if $v$ is always the largest or smallest number in $S$.
On average, the runtime lies close to the best-case time.
\begin{lemma}
  On average a fair coin needs to be tossed two times before a "heads" is seen.
\end{lemma}
\begin{proof}
  Let $E$ be the expected number of tosses before a heads is seen.
  We finish if we get heads first, else we need to repeat.
  Therefore, $E = 1 + \frac{1}{2} E$, so $E = 2$.
\end{proof}
Therefore after two split operations on average, the array will shrink to at most $3/4$ths of its size.
Let $T(n)$ be the expected running time of the algorithm on an array of size $n$.
\[T(n) \leq T(3n/4) + O(n)\]
We can then conclude that $T(n) = O(n)$.

\subsection{Matrix Multiplication}
The product of two $n \times n$ matrices $X$ and $Y$ is an $n \times n$ matrix $Z = XY$, with $(i, j)$th entry equal to $Z_{ij} = \sum_{k=1}^n X_{ik} Y_{kj}$. We can find this by taking the dot product of the $i$th row of $X$ and $j$th column of $Y$ and iterating through the number of rows in $X$ or the number of columns in $Y$. Using brute force on this problem implies a $O(n^3)$ runtime. \par
Using a divide and conquer strategy, we find that:
\[XY = \begin{bmatrix}A & B \\ C & D\end{bmatrix}\begin{bmatrix}E & F \\ G & H\end{bmatrix} = \begin{bmatrix}AE+BG & AF+BH \\ CE+DG & CF+DH\end{bmatrix}\]
Therefore we can recursively compute $AE, BG, AF, BH, CE, DG, CF, DH$, and do $O(n^2)$ additions.
The total runtime: $T(n) = 8T(n/2) + O(n^2)$, which is still $O(n^3)$.
We can further simplify this to $7$ $n/2 \times n/2$ subproblems:
\[XY = \begin{bmatrix} P_5 + P_4 - P_2 + P_6 & P_1 + P_2 \\ P_3 + P_4 & P_1 + P_5 - P_3 - P_7 \end{bmatrix}\]
where $P_1 = A(F - H), P_2 = (A+B)H, P_3 = (C+D)E, P_4 = D(G - E), P_5 = (A + D)(E + H), P_6 = (B-D)(G+H), P_7 = (A-C)(E+F)$. The new runtime for this problem is $T(n) = 7T(n/2) + O(n^2)$, which is a runtime of $O(n^{\log_2 7}) \approx O(n^{2.81})$.

\subsection{Fast Fourier Transform}
The product of two degree $d$ polynomials $C(x) = A(x) \cdot B(x) = c_0 + c_1x + \dots + c_{2d}x^{2d}$, where $c_k = a_0 b_k + a_1 b_{k-1} + \dots + a_k b_0 = \sum_{i=0}^k a_i b_{k-i}$.
\subsubsection{An Alternative Representation of Polynomials}
A degree-$d$ polynomial can either be characterized by its values at any $d+1$ distinct points, $A(x_0), A(x_1), \dots, A(x_d)$, or its coefficients, $a_0, a_1, \dots, a_d$.
To find $C(x)$ with value representation, it takes linear time. We just need to find $2d+1$ points, then at any point $z$ find the value $A(z) \times B(z)$.
We define two procedures from switching between the two representations.
\emph{Evaluation} is translating the input coefficients to values, and \emph{Interpolation} is translating the value representation to integer coefficients.

\subsubsection{Evaluation by Divide and Conquer}
If we choose $n$ points, $\{x_0, x_1, \dots, x_n\}$ to be positive and negative pairs, then the computations for finding $A(x_i) \cdot B(x_i)$ overlaps with finding $A(-x_i) \cdot B(-x_i)$. Generally, $A(x) = A_e(x^2) + x A_o(x^2)$, when we group the polynomial powers into odd or even, then factor out $x$ for the odd. Comparing the $A(x_i)$ and $A(-x_i)$ calculations, we see $A(x_i) = A_e(x_i^2) + x_i A_o(x_i^2)$ and $A(-x_i) = A_e(x_i^2) - x_i A_o(x_i^2)$.
Therefore, we can reduce the computations by only evaluating $n/2$ points at the positions $\{x_0^2, \dots, x_{n/2 - 1}^2\}$.
Each of these evaluations, $\{A_e(x_0^2), A_o(x_0^2), \dots, A_e(x_{n/2 - 1}^2), A_o(x_{n/2 - 1}^2)\}$, are also evaluation problems.
However, this trick only works at the top level, unless the $n/2$ evaluation points are also plus minus pairs for the next levels, which can only occur with complex numbers.
We need to pick $n \geq (d+1)$ points where $n = 2^k$, $k \in \ZZ$.
The initial $n$ points should be the $n$ complex solutions to $z^n = 1$, or the $n$th roots of unity.
We can easily define the $n$th roots of unity as $n$ evenly spaced points on a circle of radius $1$.
Using polar coordinates, the $n$ roots are $\omega = e^{\frac{2 \pi i}{n}}$.
This means that $\omega^{j}$ and $\omega^{j+n/2}$ are plus minus paired, since they are across from each other on the circle.
When we square the terms to evaluate $A_e(x^2)$ and $A_o(x^2)$ at $\{1, \omega^2, \omega^4, \dots, \omega^{2(n/2)-1}\}$, we get the $n/2$ points of unity.
We can think of this as if the spacing between the points in the circle of radius $1$ are now twice as far apart, meaning that the points are placed down around the circle twice, leading to two pairs of points overlapping on each other.
Putting this all together, we have the following recursive formula for the evaluation part of FFT:
\begin{enumerate}
  \item Input: $n$ coefficients that represent the polynomial, where $n$ is a power of $2$.
  \item Base case: $n = 1 \Rightarrow P(1)$. At one point we simply need to evaluate $P$ at the point.
  \item Recursive step: Call Evaluate on $P_e(x^2)$ and $P_o(x^2)$ on the $n/2$ points of unity $\{\omega^0, \omega^2, \dots \omega^{n-2}\}$. $P_e(x^2)$ has the coefficients $\{p_0, p_2, \dots, p_{n-2}\}$, and $P_o(x^2)$ has the coefficients $\{p_1, p_3, \dots, p_{n-1}\}$.
  \item From these calls, we get the evaluated terms at those points, $y_e = \{P_e(\omega^0), \dots, P_e(\omega^{n-2})\}$ and $y_o = \{P_o(\omega^0), \dots, P_o(\omega^{n-2})\}$.
  \item Now, we can calculate the evaluation at the original $n$ points, using the following equations:
    \begin{align}
      P(\omega^j) &= P_e(\omega^{2j}) + \omega^j P_o(\omega^{2j}) \\
      P(-\omega^{j}) &= P_e(\omega^{2j}) - \omega^j P_o(\omega^{2j}) = P(\omega^{j + n/2}) = P_e(\omega^{2j}) + \omega^{j + n/2} P_o(\omega^{2j}) && \text{using $-\omega^j = \omega^{j + n/2}$}
    \end{align}
  \item We can also use the following fact to clean up the equation:
    \begin{align}
      y_e[j] &= P_e(\omega^{2j}) \\
      y_o[j] &= P_o(\omega^{2j})
    \end{align}
  \item Return: We return the following evaluation at the $n$th roots of unity
    \[y = [P(\omega^0), P(\omega^1), \dots, P(\omega^{n-1})]\]
\end{enumerate}

\subsubsection{Interpolation}
We found the values are equal to FFT(coefficients, $\omega$).
It turns out that the coefficients are $\frac{1}{n}$FFT(values, $\omega^{-1}$).
The matrix representation between the two representations is the Vandermonde matrix:
\[\begin{bmatrix}A(x_0) \\ A(x_1) \\ \vdots \\ A(x_{n-1}) \end{bmatrix}
= 
\begin{bmatrix} 1 & x_0 & x_0^2 & \dots & x_0^{n-1} \\
1 & x_1 & x_1^2 & \dots & x_1^{n-1} \\
& & \vdots \\
1 & x_{n-1} & x_{n-1}^2 & \dots & x_{n-1}^{n-1}
\end{bmatrix}
\begin{bmatrix}
a_0 \\ a_1 \\ \dots \\ a_{n-1}
\end{bmatrix}\]
If $x_0, \dots, x_{n-1}$ are distinct numbers, then $M$ is invertible.
Evaluation is multiplication by $M$, while interpolation is multiplication by $M^{-1}$.
  We are able to invert the Vandermonde matrix in $O(n^2)$ instead of $O(n^3)$.
  \[M_n(\omega) = \begin{bmatrix} 1 & 1 & 1 & \dots & 1 \\
1 & \omega & \omega^2 & \dots & \omega^{2(n-1)} \\
& & \vdots \\
1 & \omega^j & \omega^{2j} & \dots & \omega^{(n-1)j} \\
& & \vdots \\
1 & \omega^{(n-1)} & \omega^{2(n-1)} & \dots & \omega^{(n-1)(n-1)}
\end{bmatrix}\]

\begin{thm}[Inversion Formula]
  $M_n(\omega)^{-1} = \frac{1}{n}M_n(\omega^{-1})$
  Because $\omega^{-1}$ is also an $n$th root of unity, interpolation or multiplication by $M_n(\omega)^{-1}$ is just an FFT operation, but with $\omega$ replaced by $\omega^{-1}$
\end{thm}
\begin{lemma}
  The columns of matrix $M$ are orthogonal to each other.
\end{lemma}
\begin{proof}
  Take the inner product of any columns $j$ and $k$ of matrix $M$.
  \[1 + \omega^{j-k} + \omega^{2(j-k)} + \dots + \omega^{(n-1)(j-k)}\]
  This is a geometric series with first term $1$, last term $\omega^{(n-1)(j-k)}$, and ratio $\omega^{(j-k)}$, which evaluates to $0$, except for when $j = k$, where the terms are all $1$ and the sum is $n$.
  Since the result is $0$, the vectors are orthogonal to each other.
\end{proof}
This means that $MM*=nI$, since $(MM*)_{ij}$ is the inner product of the $i$th and $j$th columns of $M$.
This implies $M^{-1} = (1/n) M*$.

\subsubsection{A closer look at the fast Fourier transform}
The following describes 
\begin{enumerate}
  \item Takes an input vector $a = \{a_0, \dots, a_{n-1}\}$ and complex number $\omega$ whose powers $1, \omega, \omega^2, \dots, \omega^{n-1}$ are the complex $n$th roots of unity
  \item Multiplies vector $a$ by the matrix $M_n(\omega)$
\end{enumerate}
We can split matrices using divide and conquer.
We can divide the matrix in half both ways, and split the input vector in evens and odds.
Therefore, the Product of $M_n(\omega)$ with vector $(a_0, \dots, a_{n-1})$, is the produce of $M_{n/2}(\omega^2)$ with $(a_0, a_2, \dots, a_{n-2})$ and with $(a_1, a_3, \dots, a_{n-1})$.
This divide and conquer strategy gives us a runtime of $T(n) = 2T(n/2) + O(n) = O(n \log n)$.

\pagebreak

\section{Decompositions of Graphs}

\subsection{Why graphs?}
A graph is specified by a set of vertices or nodes $V$ and edges $E$ between pairs of vertices.
An edge between $x$ and $y$ denoted by e = $\{x, y\}$ means that $x$ shares a border with $y$, and the graph is undirected.
This relationship is symmetric, so $y$ also shares a border with $x$.
An edge between $x$ and $y$ denoted by e = $(x, y)$ means that there is a directed edge going from $x$ to $y$, but not necessarily from $y$ to $x$.

\subsubsection{How is a graph represented?}
We can use an \textbf{adjacency matrix} to represent a graph.
\begin{definition}[Adjacency Matrix]
  If there are $n = |V|$ vertices $v_1, \dots, v_n$, the adjacency matrix is an $n \times n$ array whose $(i, j)$th entry is
  $$a_{ij} = \begin{cases}
    1 & \text{if there is an edge from $v_i$ to $v_j$} \\
    0 & \text{otherwise} \\
  \end{cases}$$
\end{definition}
For undirected graphs, the matrix is symmetric since both $(u, v)$ and $(v, u)$ will be equal.
Using this formula, the presence of an edge can be checked in constant time with one memory access.
However, the matrix takes up $O(n^2)$ space, which is wasteful if the graph does not have many edges. \\
An alternative representation is the \textbf{adjacency list}, which consists of $|V|$ linked lists, one per vertex.
The linked list of vertex $u$ holds the names of vertices that are connected to $u$.
Therefore the total size of the datastructure is $O(|E|)$, and checking if an edge exists is no longer constant time since it requires checking the adjacency list connected to $u$.
For undirected graphs, if $v$ is in $u$'s adjacency list, $u$ is in $v$'s adjacency list.\\

To determine which representation, adjacency matrix or adjacency list is better, we have to look at $|E|$ in relation to $|V|$.
$|E|$ can be as small as $|V|$ or as large as $|V|^2$.
If $|E|$ is large, we call this graph \textit{dense}, and it is better to use an adjacency matrix.
If $|E|$ is smaller, we call this graph \textit{sparse}, and it is better to use an adjacency list.

\subsection{Depth-first search in undirected graphs}

\subsubsection{Exploring mazes}
\textit{Depth-first search} is a linear-time procedure that determines which parts of the graph are reachable from a given vertex.
We maintain a boolean variable for each vertex indicating whether or not it has been visited already.
To return to previous junctions, we can use the stack by pushing a new vertex or poping to return to the previous one.
Instead of using a stack explicitly, we can use recursion.
The following pseudocode shows a simple depth-first search implementation.
\begin{verbatim}
procedure explore(G, v)
visited(v) = true
previsit(v)
for each edge (v, u) in E
  if not visited(u): explore(u)
postvisit(v)
\end{verbatim}
A general form of reasoning to prove graph algorithms is as follows:
\begin{itemize}
  \item{Hypothesis: For any $k \geq 0$, all nodes within $k$ hops from $v$ get visited}
  \item{Induction: Showing if all nodes $k$ hops away are visited, then so are all nodes $k+1$ hops away}
\end{itemize}
The edges that the depth-first search algorithm takes are the minimum amount of edges required to visit all the vertices connected to $v$, so they are called \textit{tree edges}.
The edges that are not visited by the algorithm are called \textit{back edges}.

\subsubsection{Depth-first search}
The previous depth-first search code only visits the portion of the graph reachable from the starting point.
To check the rest of the graph, we must call the function on unvisited vertices until all vertices have been visited.\\
To check the runtime of the algorithm, we notice that each vertex is visited only once.
During this visit, we are doing some fixed amount of work (marking the visited spot, pre/postvisit) and looping through the adjacent edges.
The total work done in the first step for all vertices is $O(|V|)$.
The loop, in total, will check over each edge, $\{x, y\} \in E$ twice, one during the call on $x$ and one during the call on $y$, making the overall time $O(|E|)$.
The total runtime of the depth-first search algorithm is therefore $O(|V| + |E|)$, which is linear in the size of its input.

\subsubsection{Connectivity in undirected graphs}
An undirected graph is \textit{connected} if there is a path between any pair of vertices.
Separate regions of vertices are each called a \textit{connected component}, since each are subgraphs that are connected.
We can adapt the depth-first search algorithm to check the connected components of a graph by assigning each node an integer identifying the connected component it belongs to.

\subsubsection{Previsit and postvisit orderings}
We can take a look at the previsit and postvisit orderings by keeping track of a clock counter, initially set to 1, and defining previsit and postvisit as the following:
\begin{verbatim}
procedure previsit(v)
pre[v] = clock
clock = clock + 1
\end{verbatim}

\begin{verbatim}
procedure postvisit(v)
post[v] = clock
clock = clock + 1
\end{verbatim}

\begin{property}
  For any nodes $u$ and $v$, the two intervals $[pre(u), post(u)]$ and $[pre(v), post(v)]$ are either disjoint or one is contained inside the other.
  We can see that the interval represents the time when a vertex is on a stack, and the last-in, first-out behavior proves the previous property.
\end{property}

\subsection{Depth-first search in directed graphs}

\subsubsection{Types of Edges}
We can run the same algorithm on directed graphs without any modifications.\\
Ancestor and descendant relationships can be read from pre and post numbers.
For edge $(u, v)$
\begin{itemize}
  \item{\textit{Tree edges} are part of the DFS forest. pre(u) < pre(v) < post(v) < post(u)}
  \item{\textit{Forward edges} lead from a node to a \textit{nonchild} descendant in the DFS tree. pre(u) < pre(v) < post(v) < post(u)}
  \item{\textit{Back edges} lead to an ancestor in the DFS tree. pre(v) < pre(u) < post(u) < post(v)}
  \item{\textit{Cross edges} lead to neither descendant nor ancestor and lead to a node that has already been completely explored. pre(v) < post(v) < pre(u) < post(u)}
\end{itemize}

\subsubsection{Directly acyclic graphs}
A \textit{cycle} in a directed graph is a circular path $v_0 \rightarrow v_1 \rightarrow \cdots \rightarrow v_k \rightarrow v_0$.
A graph without cycles is \textit{acyclic}.
\begin{property}
  A directed graph has a cycle if and only if its depth-first search reveals a back edge.
\end{property}

\begin{proof}
  One direction: If $(u, v)$ is a back edge, there is a cycle consisting of this edge together with the path from $v$ to $u$ in the search tree.\\
  Conversely: If the graph has a cycle $v_0 \rightarrow v_1 \rightarrow \cdots \rightarrow v_k \rightarrow v_0$, look at the first node discovered in the cycle, or the one with the one with the lowest pre number, $v_i$.
  All other $v_j$ of the cycle is reachable from it, and therefore will be descendants on the search tree.
  Then the edge $v_{i-1} \rightarrow v_i$ will have to lead from a node to its ancestor and is by definition a back edge.
\end{proof}

\textit{Directed Acyclic Graphs} are used to model relations like causalities, hierarchies, and temporal dependencies.
An example is many tasks with some having dependencies.
We can model each task as a node, and there is an edge from $u$ to $v$ if $u$ is a precondition for $v$.
If the graph has a cycle, there is no ordering that can work.
If the graph is a dag, we want to try and \textit{linearize} or \textit{topologically sort} the graph, or order the vertices such that each edge goes from an earlier vertex to a later vertex and all precedence constraints are satisfies.\\
We can linearize all dags by ordering them in decreasing order of their post numbers.

\begin{property}
  In a dag, every edge leads to a vertex with a lower post number.
  Otherwise, there would be a back edge that would make the graph not be a dag.
  If a dag is linearized, the vertex with the smallest post number must be a \textit{sink} (no outgoing edges), and the vertex with the highest post number must be a \textit{source} (no incoming edges).
\end{property}

\begin{property}
  Every dag has at least one source and at least one sink.
  This allows us to come up with another approach to linearization:
  \begin{enumerate}
    \item{Find a source, output it, and delete it from the graph}
    \item{Repeat until the graph is empty}
  \end{enumerate}
\end{property}

\subsection{Strongly connected components}

\subsubsection{Defining connectivity for directed graphs}
\begin{definition}
  Two nodes $u$ and $v$ of a directed graph are \textit{connected} if there is a path from $u$ to $v$ and a path from $v$ to $u$.
\end{definition}
We can then partition $V$ into disjoint sets that are called \textit{strongly connected components}.
If we shrink each set into one meta-node, and draw an edge from one meta-node to another if there is an edge between their components, the resulting \textit{meta-graph} must be a dag.
\begin{property}
  Every directed graph is a dag of its strongly connected components.
\end{property}
The connectivity of a directed graph has two tiers, the top being a dag, while the bottom being the strongly connected component within the dag's meta-nodes.

\subsubsection{An efficient algorithm}
We can depose a directed graph into its strongly connected components in linear time using depth-first search.
\begin{property}
  If the explore subroutine is started at node $u$, then it will terminate when all nodes reachable from $u$ have been visited.
  Therefore, if we call explore on a node that is in the sink meta-node of the meta-graph, we will retrieve exactly that component.
\end{property}
\begin{property}
  The node that receives the highest post number in a depth-first search must lie in a source strongly connected component.
\end{property}
\begin{property}
  If $C$ and $C'$ are strongly connected components, and there is an edge from a node in $C$ to a node in $C'$, the highest post number in $C$ is bigger than the highest post number in $C'$.
\end{property}
\begin{proof}
  We have to consider two cases of the depth-first search, based on which component it visits first.
  If it visits component $C$ before $C'$, all of $C$ and $C'$ will be traversed, and the first node in $C$ that is visited will have a higher post number than any node of $C'$.
  If $C'$ gets visited first, then the depth-first search will be stuck after seeing $C'$ but before seeing any of $C$.
\end{proof}

We can then find the sink meta-node by reversing graph $G$ to create $G^R$, then finding the node with the highest post number in $G^R$.
It's strongly connected component would then be the sink strongly component in $G$.
We can then keep deleting the sink in $G^R$ to successively output the strongly connected components.\\
The final algorithm (which runs in linear time) looks like this:
\begin{enumerate}
  \item{Run depth-first search on $G^R$}
  \item{Run the undirected connected components algorithm on $G$, and during the depth-first search, process the vertices in decreasing order of their post numbers from step 1}
\end{enumerate}

\pagebreak

\section{Paths in Graphs}

\subsection{Distances}
The \textit{distance} between two nodes is the length of the shortest path between them.
We can imagine the distance from a node by imagining the paths as strings and nodes as balls connected to the string.
The distance between nodes is how far they hang down when you lift them up.

\subsection{Breadth-first Search}
We can see that lifting the nodes partitions the graphs into layers based on distance from the starting node.
\begin{definition}[Breadth-First Search]
Breadth first search traverses the nodes by layer, since it is easy to get the nodes at distance $d+1$ if we have all the nodes at distance $d$.
\begin{enumerate}
  \item{Initially, the queue $Q$ consists of only $s$, the one node at distance $0$}
  \item{For each subsequent distance $d = 1, 2, 3, \dots$, there is a point in where $Q$ contains all the nodes at distance $d$ and nothing else.}
  \item{As these nodes are processed and ejected, their unvisited neighbors at position $d+1$ are injected to the end of the queue.}
\end{enumerate}

\begin{verbatim}
  for all u in V:
    dist[u] = inf

  dist(s) = 0
  Q = [s]
  while Q is not empty:
    u = eject(Q)
    for all edges (u, v) in E:
      if dist(v) = inf:
        inject(Q, v)
        dist(v) = dist(u) + 1
\end{verbatim}
\end{definition}

The overall runtime is $O(|V| + |E|)$, since each vertex is pushed and popped from the queue once, making $2|V|$ queue operations.
For each vertex, the loop looks at each edge once in direct graphs or twice in undirected graphs, taking $O(|E|)$ time.

\subsection{Lengths on Edges}
Breadth-first search treats all edges as having the same length.
In general, however, each edge $e \in E$ has a length $l_e$. If $e = (u, v)$, we will also write $l(u, v)$ or $l_{uv}$.

\subsection{Dijkstra's Algorithm}

\subsubsection{An Adaptation to BFS}
One solution to weighted graphs is to break a graph $G$'s edges into unit-length pieces through "dummy" nodes.
This is done by replacing an edge $e = (u, v)$ of $E$ by adding $l_e - 1$ dummy nodes between $u$ and $v$, producing a graph $G'$.
We can then run BFS on $G'$.
One issue with this approach is if $G$ has very long edges.\\
We can solve this problem with alarms, with estimated times of arrival.
When we actually reach a node or an alarm clock rings, we can update the remaining alarm clocks.
We can implement this system by using a priority queue, which allows us to tell which alarm will go off next. \\
\begin{verbatim}
  for all u in V:
    dist(u) = inf
    prev(u) = nil
  dist(s) = 0

  H = makequeue(V)
  while H is not empty:
    u = deletemin(H)
    for all edges (u, v) in E:
      if dist(v) > dist(u) + l(u, v):
        dist(v) = dist(u) = l(u, v)
        prev(v) = u
        decreasekey(H, v)
\end{verbatim}

\subsubsection{An Alternative Derivation}
We can compute the shortest paths in using a different method.
We start from the point $s$, then consider the closest nodes and moving on to those further away.
When the "known region" is some subsets of vertices $R$ that includes $s$, the next addition should be the node outside $R$ closest to $s$.
To find the node $v$ closest to $s$ and outside $R$, we first consider $u$, the closest vertex to $v$ in $R$.
$u$ must be closer to $s$ than $v$ is, assuming that all edge lengths are positive.
So the shortest path from $s$ to $v$ is a known shortest path extended by a single edge.
We can thus find the path to $v$ by trying all the single-edge extensions from the currently known shortest paths, finding the shortest such path, then adding it and the endpoint to $R$.

\subsubsection{Running Time}
Dijkstra's algorithm is structurally identical to breadth-first search, but the priority queue operations take logarithmic time.
There are at most $|V|$ insert operations, $|V|$ deletemin and $|V|+|E|$ insert/decreasekey oeprations. This means that total time for Dijkstra's algorithm varies based on the priority queue operation used.
\begin{center}
\begin{tabular}{| c | c | c | c |}
  \hline
  Implementation & deletemin & insert/deletekey & overall \\
  \hline
  Array & $O(|V|)$ & $O(1)$ & $O(|V|^2)$ \\
  Binary Heap & $O(\log|V|)$ & $O(\log|V|)$ & $O((|V| + |E|) \log |V|)$ \\
  $d$-ary heap & $O(\frac{d \log |V|}{\log d})$ & $(\frac{\log |V|}{\log d})$ & $O((|V| \cdot d + |E|) \frac{\log |V|}{\log d})$ \\
  Fibonacci Heap & $O(|V|)$ & $O(1)$ & $O(|V| \log |V| + |E|)$ \\
  \hline
\end{tabular}
\end{center}
To determine which implementation is best, we need to consider if a graph is \textit{sparse} or \textit{dense}.
For all graphs, $|E|$ is less than $|V|^2$.
If it is $\Omega(|V|^2)$, the array implementation is faster.
A binary heap becomes preferable if $|E|$ dips below $|V|^2 / \log|V|$. \\
A $d$-ary heap is a generalization of a binary heap (which has $d = 2$).
The optimal choice is $d \approx |E| / |V|$, or the average degree of the graph.
For sparse graphs ($|E| = O(|V|)$), the runtime is $O(|V| \log |V|)$.
For dense graphs ($|E| = \Omega(|V|^2)$), the runtime is $O(|V|^2)$.
For intermediate density graphs ($|E| = |V|^{1 + \delta}$), the runtime is $O(|E|)$. \\
The Fibonacci heap is not touched upon much, except that the insert operation takes varying amounts of time that averages at $O(1)$, so the \textit{amortized} cost of heap insert is $O(1)$.

\subsection{Priority queue implementations}

\subsubsection{Array}
We can implement a priority queue as an unordered array of key values for all elements, with the initial values set to $\infty$.
To insert or decreasekey, we only need to change a key value, taking $O(1)$ time.
To deletemin, we have to perform a scan of the list, taking $O(n)$ time.

\subsubsection{Binary Heap}
In the Binary Heap, elements are stored in a complete binary tree.
We enforce the constraint that the key value of each node is less than or equal to its children.
Therefore, the root contains the smallest element. \\
To insert, we place a new element at the bottom and "bubble up" by comparing to its parent and swapping the two if it is smaller.
The number of swaps is at most the height of the tree, which is $\log_2 n$ with $n$ elements.
A decreasekey does the same, except it bubbles up an element already inside the tree from its current position.
To deletemin, we return the root value and replace it with the last node in the tree, also the rightmost node in the bottom row of the tree.\\
We can represent the binary tree using an array.
The node at position $j$ in the array will have its parent in position $\floor{\frac{j}{2}}$ and children in position $2j$ and $2j + 1$.

\subsubsection{$d$-ary heap}
A $d$-ary heap is similar to a binary heap, except each node has $d$ children instead of $2$.
The height of the tree is therefore $\Theta(\log_d n) = \Theta ((\log n)) / (\log d)$.
Inserts are therefore sped up to $\Theta(\log d)$.
However, deletemin operations are now $O(d \log_d n)$. \\
To represent a $d$-ary heap as an array, the node at position $j$ on the array has parent $\left\lceil \frac{j-1}{d} \right\rceil$ and children $\{(j-1)d + 2, \ldots, \min\{n, (j-1)d + d + 1\}\}$.

\subsection{Shortest paths in the presence of negative edges}
\subsubsection{Negative edges}
Dijkstra's algorithm assumes that the shortest path from the starting point $s$ to any node $v$ must pass through nodes that are closer than $v$.
This assumption no longer holds when the edge weights are negative, since a negative edge weight from a node farther than $v$ to $s$ could be the shortest.\\
To account for this, we can update all the edges $|V| - 1$ times, by checking the connected edges and getting the min distance.
This is called the Bellman-Ford algorithm, which takes a runtime of $O(|V| \cdot |E|)$.

\subsubsection{Negative cycles}
If there is a negative cycle, it would not make sense to ask the shortest path.
Going around the cycle causes the shortest path to continue decreasing.
We can detect negative cycles by performing the Bellman-Ford algorithm one more time.
If some dist value changes during the last cycle, we know there must be a negative edge.

\subsection{Shortest Paths in DAGs}
We now take a look at DAGs, which are graphs without cycles, and therefore 
We observe that in any path of a dag, the vertices appear in increasing linearized order.
We can therefore topologically sort the dag by depth-first search, then visit the vertices in the sorted order and update the edges of each.

\pagebreak
\section{Greedy Algorithms}
Greedy algorithms build up a solution by always choosing the next piece that offers the most obvious and immediate benefit.
This will only work on certain tasks, where thinking ahead is not needed.

\subsection{Minimum Spanning Trees}
Suppose you have to network a collection of computers by linking pairs of them.
Each edge on the graph represents a possible connection, and the edge weight represents the maintainance cost.
\begin{property}
  Removing a cycle edge cannot disconnect a graph.
\end{property}
This means that the solution should be connected and acyclic (or a tree), since if the graph was cyclic removing an edge would reduce the cost.
These types of trees are called minimum spanning trees. \\
\textit{Input:} An undirected graph $G = (V, E)$ with edge weights $w_e$.
\textit{Output:} A tree $T = (V, E')$ with $E' \subseteq E$ that minimizes $$\text{weight}(T) = \sum_{e \in E'} w_e$$.

\subsubsection{A Greedy Approach}
Kruskal's minimum spanning tree starts with an empty graph, then selects edges from $E$.
It repeatedly adds the next lightest edge that doesn't produce a cycle.
\begin{property} 
  A tree on $n$ nodes has $n-1$ edges.\\
  This can be seen by building the tree one edge at a time.
  The first node doesn't need an edge, but every other node requires an edge to connect to the main tree.
\end{property}
\begin{property}
  Any connected, undirected graph $G = (V, E)$ with $|E| = |V| - 1$ is a tree.\\
  We simply need to prove that $G$ is acyclic, then it will satisfy the definition of a tree.
  We can run the following iterative procedure: while the graph contains a cycle, remove one edge from the cycle.
  The process terminates when some graph $G' = (V, E')$, $E' \subseteq E$, which is acyclic and connected, so $G'$ is a tree.
  However, we can see $|E'| = |V| - 1$ by the previous property.
  This means that $E' = E$, $G' = G$, no edges were removed, and $G$ was acyclic to start with.
\end{property}
\begin{property}
  An undirected graph is a tree if and only if there is a unique path between any pair of nodes.\\
  First we solve the forward direction.
  In a tree, any two nodes can only have one path between them.
  If there were two paths, the union of these paths would contain a cycle.
  For the backwards direction, we note that if a graph has a path between any two nodes, then it is connected.
  If these paths are unique, the graph is also acyclic, since a cycle would require two paths between a pair of nodes.
\end{property}

\subsubsection{The cut property}
\begin{property}[Cut Property]
  Suppose edges $X$ are part of a minimum spanning tree of $G = (V, E)$.
  Pick any subset of nodes $S$ for which $X$ does not cross between $S$ and $V-S$.
  Then $X \cup \{e\}$ is part of some MST.
\end{property}

A cut is any partition of vertices into two groups, $S$ and $V - S$.
The property states that it is safe to add the lightest edge across any cut, provided $X$ has no edges across the cut.
To see why the property holds, we can assume that edges $X$ are a part of some MST $T$.
Let's assume that $e$ is not in $T$.
We can construct a different MST $T'$ containing $X \cup \{e\}$ by altering $T$ and changing one of its edges.
We do this by adding edge $e$ to $T$.
Since $T$ is connected, it already has a path between the endpoints of $e$.
Therefore, it must have another edge $e'$ across the cut $(S, V - S)$.
If we remove this edge, we are left with $T' = T \cup \{e\} - \{e'\}$.
Since $T'$ is connected, has the same number of vertices and edges as $T$, then $T'$ is a MST.
If we compare the two tree weights, we get that
$$\text{weight}(T') = \text{weight}(T) + w(e) - w(e')$$
Since both $e$ and $e'$ are edges across $S$ and $V - S$, and we assumed $e$ is the lightest edge, $w(e) \leq w(e')$, therefore $\text{weight}(T') \leq \text{weight}(T)$.
Since $T$ is an MST, $w(e) = w(e')$.

\subsubsection{Kruskal's Algorithm}
To justify Kruskal's algorithm, we notice that at any moment the edges already chosen form a partial solution, or a group of connected components part of the final solution.
The next edge to be added, $e$, should connect two components, $T_1$ and $T_2$.
Since $e$ is the lightest edge that doesn't produce a cycle, it satisfies the cut property.\\
In order to test if a candidate edge $u - v$ should be added, we also need to check if $u$ and $v$ lie in different components.
We can model the algorithm's state using a collection of disjoint sets, each of which contains the nodes of a component.\\
Initially, each node is a component by itself.
We repeatedly test pairs of nodes to see if they belong to the same set.
Whenever we add an edge, we merge the two components of the nodes.\\
The final algorithm uses $|V|$ makeset, $2|E|$ find, and $|V| - 1$ union operations.

\subsubsection{A Data Structure for Disjoint Sets}
\paragraph{Union by rank}
We can store sets as a directed tree, where nodes of the tree are elements of the set, and each have parents that eventually lead to the root of the tree.
The root element is a representative for the entire set, with its parent pointer being a self-loop.
Each node also has a rank that can be interpreted as the height of the subtree hanging from the node.
When using the union operation, we should make sure to keep the trees shallow by making the root of the shorter tree point to the root of the taller tree.
This will ensure that the height of the tree will only increase if the two trees are equally tall.
Instead of using height, we can use the rank of the tree, which is why the algorithm is called \textit{union by rank}.

\begin{property}
  For any $x$, $\text{rank}(x) < \text{rank}(\pi (x))$
\end{property}

\begin{property}
  Any root node of rank $k$ has at least $2^k$ nodes in its tree.
\end{property}

\begin{property}
  If there are $n$ elements overall, there can be at most $\frac{n}{2^k}$ nodes of rank $k$.
  This means that the maximum rank is $\log n$.
\end{property}

\paragraph{Path Compression}
The total time for Kruskal's algorithm becomes $O(|E| \log |V|)$ for sorting the edges, with another $O(|E| \log |V|)$ for the union and find operations.
We can further improve the performance of the algorithm by using path compression.
While following the pointer to the root of the tree, we can change all the pointers to point directly to the tree.
This turns the average time per operation and the amortized cost to be $O(1)$ from $O(\log n)$.

\paragraph{A randomized algorithm for minimum cut}
Let's define a cut $(S, \overline{S})$ in the graph.
Suppose the graph we are working with is unweighted, and the edges are ordered uniformly at random.
If we define the size of a cut $(S, \overline{S} )$ as the number of edges crossing between $S$ and $\overline{S} $, there is at least a $\frac{1}{n^2}$ probability that the random cut is the minimum cut in the graph.
This means that repeating the process $O(n^2)$ times and outputting the smallest cut gives a high probability of finding the smallest cut.
We then have an $O(mn^2 \log n)$ algorithm for unweighted minimum cuts, which can be reduced to $O(n^2 \log n)$.\\
To see why the probability for a minimum cut at each iteration is at least $\frac{1}{n^2}$, we can check Kruskal's algorithm.
At any stage of the algorithm, the vertex set $V$ is partitioned into connected components.
The only edges we can add to the tree have endpoints in separate components.
Let's call the size of the minimum cut $C$.
The number of edges incident to each component is at least $C$.
So, if there are $k$ components in the graph, there will be at least $\frac{kC}{2}$ number of eligible edges that can be cut.
The chance that the next eligible edge is from the minimum cut is at most $\frac{C}{(\frac{kC}{2})} = \frac{2}{k}$.
Thus, the probability of leaving the minimum cut intact is at least $\frac{k-2}{k}$, and the chance that Kruskal's algorithm leaves the minimum cut intact until the last edge is at least $$\frac{n-2}{n} \cdot \frac{n-3}{n-1} \cdot \frac{n-4}{n-2} \cdots \frac{2}{4} \cdot \frac{1}{3} = \frac{1}{n(n-1)}$$

\subsubsection{Prim's Algorithm}
In Prim's algorithm, an intermediate set of edges $X$ forms a subtree, where $S$ is the set of this tree's vertices.
On each iteration, $X$ grows by one edge, which is the lightest edge between a vertex in $S$ and a vertex outside $S$.
This is similar to Dijkstra's algorithm, except that the value of a node is weight of the lightest incoming edge from set $S$, while in Dijkstra's the weight is the path from the start node.

\subsection{Huffman Encoding}
In an MP3 audio compression, the sound is encoded in three steps:
\begin{enumerate}
  \item It is digitized by sampling at regular intervals, translating to a sequence of real numbers $s_1, s_2, \ldots s_T$.
  \item Each sample $s_t$ is quantized, or approximated by a nearby number from a finite set $\Gamma$, which is carefully chosen to be indistinguishable to the human ear.
  \item The resulting string of length $T$ over alphabet $\Gamma $ is encoded in binary.
\end{enumerate}

Let's focus on the last step.
For a simple example, assume that $T$ is 130 million and $\Gamma $ only has four values, $A, B, C, D$.
We might choose to encode with 2 bits per symbol, with $00$ for $A$, $01$ for $B$, $10$ for $C$, and $11$ for $D$, requiring 260 megabits.\\
However, let's say that the four symbols are not equally abundant.
\begin{center}
\begin{tabular}{c c}
  Symbol & Frequency \\
  \hline
  A & 70 million\\
  B & 3 million\\
  C & 20 million\\
  D & 37 million\\
\end{tabular}
\end{center}
We may then consider a \textit{variable-length} encoding, where just one bit is used for frequent characters at the expense of three or more for other characters.
In order to ensure uniqueness, we need to insist on a \textit{prefix-free} property, where no codeword can be a prefix of another codeword.
We can represent any prefix-free encoding with a full binary tree, where every node has either zero or two children, with the leaves as symbols and codeword generated by a path from a root to a leaf.\\
We want to find the optimal coding tree, given the frequencies, $f_1, f_2, \ldots f_n$ of $n$ symbols.
Then we want to minimize the overall length of the encoding, $$\text{cost of tree} = \sum_{i=1}^{n} f_i \cdot (\text{depth of $i$th symbol in tree})$$
Since we are representing as a tree, we can define each internal node to be the sum of the frequencies of its descendant leaves, or how many times the internal node is visited during encoding or decoding.
During encoding, every time we visit a nonroot internal node, one bit gets output.
Thus, another useful way of writing the cost function is the sum of the frequencies of all leaves and internal nodes, except the root.\\
The two symbols with the lowest frequencies should be at the bottom of an optimal tree, otherwise swapping them with symbols with a lower frequency would improve the encoding.
This suggests us to use a greedy algorithm: find the two symbols with the smallest frequencies, $i$ and $j$, make them the children of a new node which then has frequency $f_i + f_j$.
Then, the cost of the tree is cost $f_1 + f_2$ plus the cost of a tree with $n-1$ leaves of frequencies $(f_1 + f_2), f_3, f_4, \ldots, f_n$.
The resulting algorithm can make use of a priority queue, and takes $O(n \log n)$ time if a binary heap.

\paragraph{Entropy}
Let's say we have a horse race with four horses.
We record the probability of them getting first, second, third, or other in their previous races.
By determining the total number of bits required to encode the horse track records, we can determine how compressible and therefore predictable the horses are.\\
In general, suppose there are $n$ possible outcomes with probabilities $p_1, p_2, \ldots p_n$.
If a sequence of $m$ values is drawn from the distribution, the $i$th outcome will pop up roughly $mp_i$ times.
For simplicity, assume that the $p_i$'s are all powers of 2, or in the form $\frac{1}{2^k}$.\\
By induction, the number of bits needed to encode the sequence is $$\sum_{i=1}^n mp_i \log (\frac{1}{p_i})$$
The average number of bits needed to encode a single draw from the distribution is $$\sum_{i=1}^n p_i \log (\frac{1}{p_i})$$
This is the \textit{entropy} of the distribution, or a measure of how much randomness it contains.

\subsection{Horn Formulas}
Horn formulas are a framework for computers to express logical facts and derive conclusions.
The most primitive object is a \textit{boolean}, which is either true or false.
A \textit{literal} is a variable $x$ or its negation $\overline{x} $.
There are two types of \textit{clauses}:
\begin{enumerate}
  \item \textit{Implications}: A left-hand side of an AND of any number of positive literals and right-hand side is a single positive literal. $$(z \land w) \implies u$$
  \item \textit{Negative clauses}: An or of any number of negative literals. $$(\overline{u} \lor \overline{v} \lor \overline{y})$$
\end{enumerate}
Given a set of clauses, we want to determine if there is a set of some variables that satisfies all the clauses, or a \textit{satisfying assignment}.\\
One strategy is to start with all the variables being false, then set some of them to true only if we absolutely have to or else an \textit{implication} would be violated.
Once we are done and all the \textit{implications} are satisfied, we then turn to the \textit{negative clauses} and make sure they are satisfied as well.
This follows a greedy scheme:
\begin{verbatim}
  Input: A Horn formula
  Output: A satisfying assignment, if one exists
  set all variables to false
  while there is an implication that is not satified:
    set the right-hand variable of the implication to true

  if all pure negative clauses are satisfied: return the assignment
  else: return ``formula is not satisfiable``
\end{verbatim}
We have to convince ourselves that if the algorithm finds no satisfying assignment, there really is none.
This is satisfied by the fact we are being "stingy," which maintains the invariant that:
\begin{center}
  If a certain set of variables is set to true, then they must be true in any satisfying assignment.
\end{center}
Therefore, if the truth assignment after the while loop does not satisfy the negative clauses, there can be no satisfying truth assignment.

\subsection{Set Cover}
Set cover takes in a set of elements $B$ and sets $S_1, \ldots, S_m \subseteq B$.
Then, it outputs the selection of $S_i$ whose union is $B$, and the number of sets picked, or the cost.
This problem seems to have a greedy solution, where we pick the set $S_i$ with the largest number of uncovered elements each time.
However, this greedy scheme is not optimal.
\paragraph{Claim}
Suppose $B$ contains $n$ elements that the optimal cover consists of $k$ sets.
Then the greedy algorithm will use at most $k \ln n$ sets.\\
Let $n_t$ be the number of elements not covered after $t$ iterations of the greedy algorithm.
Since these remaining elements are covered by the optimal $k$ sets, there must be some set with at least $n_t / k$ of them.
Therefore, the greedy strategy will ensure that $$n_{t+1} \leq n_t - \frac{n_t}{k} = n_t(1 - \frac{1}{k})$$
By repeated application, this implies $n_t \leq n_0 (1 - \frac{1}{k})^t$.
We can use the following inequality: $$1 - x \leq e^{-x} \text{for all $x$, with equality if and only if $x = 0$}$$
Thus: $$n_t \leq n_0(1 - \frac{1}{k})^t < n_0(e^{-\frac{1}{k}})^t = ne^{-\frac{t}{k}}$$
At $t = k \ln n$, $n_t$ is less than $ne^{-\ln n} = 1$, meaning no elements remain to be covered.\\
The ratio between the greedy solution and optimal solution is always less than $\ln n$, which is called the \textit{approximation factor}.

\pagebreak
\section{Dynamic Programming}

\end{document}
